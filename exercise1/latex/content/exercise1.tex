\section{Pitfalls of integer and floating point arithmetic}

The aim of the first exercise is to illustrate some pitfalls one can encounter when dealing
with in integer and floating point arithmetic in C and C++.\\
For this purpos some code samples are provided which should be executed and the results 
explained.

\subsection*{1. Sample}
The first code sample shows the difference between integer and floating point devision.
\begin{lstlisting}[language=C++]
    int   i = 7;
    float y = 2*(i/2);
    float z = 2*(i/2.);
    printf("%e %e \n", y,z); 
\end{lstlisting}
If executed this code returns the following numbers to the console 
for $x$ and $y$.

\begin{lstlisting}[language=C++]
    y = 6.000000e+00
    x = 7.000000e+00
\end{lstlisting}

Clearly the first number is not the expected result. The reason for this is that the calculation of 
$\textnormal{i}/2$ is done as an integer devision. This devision cuts of the floating part of the number.
So the calculation $7/2$ gives the result 3. This is then multiplied by 2 and cast into a
floating point number. But if one of the numbers of the devision is marked as a floating 
point number like $7/2.$ then the correct result is calculated.

\subsection*{2. Sample}
\begin{lstlisting}[language=C++]
    double a = 1.0e17;
    double b = -1.0e17;
    double c = 1.0;
    double x = (a + b) + c;
    double y = a + (b + c);

\end{lstlisting}

If this second code sample is executed the following 
result is printed to the console for $x$ and $y$.

\begin{lstlisting}[language=C++]
    x = 1.000000e+00
    y = 0.000000e+00 
    
\end{lstlisting}

Again it is clear that only the result for $x$ is correct.
The problem with the second part is the summation of $b$
and $c$ which are numbers of very different magnitude.
When adding the numbers the small value will be converted
into a representation with the same exponent as the 
larg number. But the mantissa of the floating point number
can only cover a certain range and 1.0 falls beneath the 
machine precision. \\
As a result c is not significant in the addition of the small
and the large number and therefor only the large number is returned 
which leads to an incorrect final result.
This problem does not occure if the two large numbers are 
substracted first.\\
So the reason why the law of associativity is broken in this
example is that computers can only add numbers with the same
exponent and that the mantissa of a floating point number
can only cover numbers in a certain range.\\


\subsection*{3. Sample}

\begin{lstlisting}[language=C++]
    float x = 1e20;
    float y;
    y = x * x;
    printf("%e %e\n", x,y/x);

\end{lstlisting}

The result of the last sample code is:
\begin{lstlisting}[language=C++]
    x = 1.000000e+20 
    y/x = inf
    
\end{lstlisting}

The problem in this case is that the maximum number that
can be represented with a float is approximately $3,4\textrm{e}38$
and therefore $y=1\textrm{e}40$ will actually be represented as infinity.
This is the reason why the devision $y/x$ also gives infinity as a result.\\